%!TEX root = P165_notes.tex

\documentclass[12pt]{article}
\newcommand{\FlipTR}{UCR-TR-2022-FLIP-00X}
\input{FlipPreamble}
\input{FlipAdditionalHeader} %% Use this define additional macros
%!TEX root = P165_notes.tex
%% These are packages that need to be called at the end of the preamble 
%% or else they may lead to potential package conflicts.

%%%%%%%%%%%%%%%%%%%
%%%  HYPERREF  %%%%
%%%%%%%%%%%%%%%%%%%

%% This package has to be at the end; can lead to conflicts
\usepackage[
	colorlinks=true,
	citecolor=green!50!black,
	linkcolor=NavyBlue!75!black,
	urlcolor=green!50!black,
	hypertexnames=false]{hyperref}

% Leo's ORCID link
% https://github.com/duetosymmetry/orcidlink-LaTeX-command
\usepackage{orcidlink}

\begin{document}

%\thispagestyle{empty}		% If no TR#
\thispagestyle{firststyle} 	% TR# on 1st page (pdfsync may fail on 1st page)

\begin{center}
	
	{\large {P165:}}
    {\Large \bf Introduction to Particle Physics}

    \vskip .7cm

%% SINGLE AUTHOR FORMAT
%% --------------------
	\textbf{Flip Tanedo} \\
	\texttt{\footnotesize \email{flip.tanedo@ucr.edu}}

	\vspace{-1em}
    \begin{institutions}[1.5cm]
    \footnotesize
    {\it 
	    Department of Physics \& Astronomy, 
	    University of  California, Riverside, 
	    {CA} 92521	    
	    }    
    \end{institutions}


\end{center}




%%%%%%%%%%%%%%%%%%%%%
%%%  ABSTRACT    %%%%
%%%%%%%%%%%%%%%%%%%%%

\begin{abstract}
\noindent 
This is a 10 week course on the theory of the Standard Model geared towards undergraduates who have taken at least the first of a three-quarter sequence in quantum mechanics. 
%
Last Compiled: \today
\end{abstract}



\small
\setcounter{tocdepth}{2}
\tableofcontents
\normalsize
%\clearpage


%%%%%%%%%%%%%%%%%%%%%
%%%  THE CONTENT  %%%
%%%%%%%%%%%%%%%%%%%%%

\section{Introduction}

To be filled in. 

\section{Units and Natural Units}

Imagine that you have three apples. This is a number (three) an a unit (apple). The meaning of the unit depends on what you're using it to measure. For example, if apples are \$1 each, then you could use an apple as a unit of currency. The way to do this is to simply \emph{multiply by one}:
\begin{align}
  (3\text{ apples}) \times \left(\frac{\text{\$ 1}}{\text{apple}}\right)
  &= \$ 3 \ .
\end{align}
We have used the fact that the exchange rate is simply the statement that
\begin{align}
  1\text{ apple} &= \$1
  & \Rightarrow &&
  1 &= \frac{\$ 1}{1\text{ apple}} \ .
\end{align}
% You can do a similar thing for [kilo-]calories or any other conversion rate. 


\begin{exercise} An apple is also roughly 100 dietary calories. If a friend buys me a coffee for \$2, does this mean that I can pay that friend back in 200 dietary calories? Comment on the relative merits of using apples as an exchange between monetary value and dietary energy.
\end{exercise}

\begin{exercise}
At the time of this writing, one dollar is worth three-fourths of a British pound sterling: $\$1 = \textsterling0.75$. Write an expression that is equal to one that is a ratio of pound sterling to dollars. Use this ratio to convert the cost of two apples in the \acro{UK}. Is the exchange rate a good conversion factor? What if a friend of mine bought me coffee in Riverside, and I promised to pay them back in pound sterling in two years? 
\end{exercise}


All that matters is that the conversion is constant. Indeed, the constants of nature make very good `exchange rates.' For example, in particle physics we like to use \textbf{natural units}. This is the curious statement that
\begin{align}
  \hbar = c = 1 \ .
\end{align}
At face value, this doesn’t make sense. $\hbar$ has units of action, $c$ is a speed, and 1 is dimensionless. However, because nature gives us a \emph{fundamental} unit of action and a \emph{fundamental} unit of speed, we may use them as conversion factors (exchange rates),
\begin{align}
  c = 3 \times 10^{10}~\text{cm}/\text{s} \ .
\end{align}
If $c=1$, then this means
\begin{align}
  1 \text{ s} &=  3 \times 10^{10}~\text{cm} \ .
\end{align}
This, in turn, connects a unit of time to a unit of distance. By measuring time, the constant $c$ automatically gives us an associated distance. The physical relevance of the distance is tied to the nature of the fundamental constant: one second (or `light-second') is the distance that a photon travels in one second. Observe that this only works because $c$ is a constant. 





 \section*{Acknowledgments}


\acro{PT}\ thanks the Winter 2018, Winter 2020, and Spring 2022 cohorts of students for their patience as this course was developed. 
%
\acro{PT} thanks the Aspen Center for Physics (\acro{NSF} grant \#1066293) for its hospitality during a period where part of this work was completed. 
%
% \acro{PT} is supported by the \acro{DOE} grant \acro{DE-SC}/0008541.
\acro{PT} is supported by a \acro{NSF CAREER} award.

%% Appendices
% \appendix


%% Bibliography
%\bibliographystyle{utcaps} 	% arXiv hyperlinks, preserves caps in title
%\bibliographystyle{utphys} 	% arXiv hyperlinks
% \bibliography{bib title without .bib}


\end{document}